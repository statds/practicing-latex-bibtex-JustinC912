\documentclass[12pt, letterpaper]{article}

\begin{document}

% making title
\title{The Best Academic Paper Ever}

\author{Justin Chan}
Jun Yan
Department of Statistics, University of Connecticut

\begin{abstract}
The title of being the "Best Academic Paper Ever" is widely debated amongst experts but ever since Justin Chan pulled up to the block, his papers have given new meaning to the meaning of what it takes to be the "Best Academic Paper." By servicing free oreos, milk, and magnolia pumpkin spice pudding to the readers of his paper, the landscape of what makes a good academic paper special has forever been changed.

\end{abstract}

\section{Introduction}
OREOS MILK AND MAGNOLIA PUMPKIN SPICE PUDDING yesssssir

The rest of the paper is organized as follows. Section~\ref{sec:oreos} investigates why oreos are good 
for the soul in the scenario where you are eating them. Section~\ref{sec:milk} investigates why milk is 
beneficial for your body, through providing various nutrients to promote vitamins and bone health. 
Section~\ref{sec:pudding} explores the case of pudding in your refrigerator and makes the argument 
that you should eat it while reading this academic paper. Finally, Section~\ref{sec:conclusion} makes 
the bold claim that a jar of sugar a day keeps the doctor away, for good.

\section{Oreos are Good for the Soul} \label{sec:oreos}

The creamy part of an oreo is white. Your soul is white (I think). Boom.

\section{Milk is Good for the Body} \label{sec:milk}

Milk is white. Your bones are white. Boom.

\section{Pudding} \label{sec:pudding}

Pudding.

\section{Conclusion} \label{sec:conclusion}

Conclusion

\appendix

\section{Appendix}

\bibliographystyle{asa}
\bibliography{citations}
